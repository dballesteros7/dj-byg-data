\documentclass[11pt,a4paper,twoside]{scrartcl}
\usepackage[utf8]{inputenc}
\usepackage[english]{babel}
\usepackage{amsmath}
\usepackage{amsfonts}
\usepackage{amssymb}
\usepackage{graphicx}
\usepackage{hyperref}

\bibliographystyle{acm}
  
\author{Diego Ballesteros, Tracjhe Kralev, Tribhuvanesh Orekondy}
\title{DJ Tini Data}
\subtitle{The small scale implementation of DJ Byg Data}
\date{\today}

\begin{document}
  \maketitle
  \section{Introduction}
    The general aim of this project is to develop a system able to produce
    information out of data extracted from songs. This is the general
    aim of the research field known as Music Information Retrieval (MIR),
    this field has been active for quite some time \cite{nameThatTune:1993}
    and its applications can be seen in several commercial products such as
    Pandora or Spotify \cite{recommendation:2010}.

    A great portion of the activity in MIR deals with song meta-data and audio
    features \cite{McFee:2012:MSD:2187980.2188222}. However, in this project we
    focus on lyrics as the main source of data.
    The main task that we pursue is to answer the following question:
    
    "Is it possible to determine the genre of a song based solely on its
    lyrics?"
     
    In order to accomplish this, we first explored how to extract genre
    information from the songs' lyrics and then evaluated the quality of the
    judgments. Lastly we made the developed methods available as a web
    application where arbitrary lyrics can be input in order to obtain
    a genre prediction.
    
    The report is organized as follows, first we present the input datasets used
    for the task and their respective pre-processing and manipulation. We then
    proceed to briefly explain the computing and analysis techniques chosen for
    the task. In the third section we present details of the implementation of
    these techniques. Finally the obtained results and the performance of the
    algorithms is discussed.
  \section{Data}
  
    \subsection{Description} \label{sec:data:subsec:description}
    In order to carry out the previously stated task, we selected the Million
    Songs Dataset (MSD) \cite{Bertin-Mahieux2011}.
    This dataset is a free collection of audio features and metadata for a million
    of tracks, it was collected by the LabROSA at Columbia University using
    the EchoNest API
    \footnote{A music intelligence platform - \url{http://the.echonest.com/}}.
    
    However, the main dataset contains only song metadata
    (e.g. artist and album) and audio features which are not of interest
    for this project. Fortunately, the dataset is accompanied by four companion
    datasets:
    
    \begin{description}
      \item[SecondHandSongs] List of cover songs within the MSD.
      \item[Taste profile] User data from EchoNest for a subset of the MSD.
      \item[musiXmatch] Lyrics for the songs in the MSD
                        \footnote{Lyrics are available only in bag of word
                                  representation due to copyright.}.
      \item[Last.fm] Tags and similarity information for songs in the MSD.
    \end{description}
    
    Of interest for the current task are the last two datasets. The musiXmatch
    dataset provides the lyrics needed for the analysis and the Last.fm tags can
    be used to identify the genre of the songs.
    
    Summarizing, the data to be used is:
    
    \begin{itemize}
      \item The main MSD, only the IDs are used. There are 1M in the
            dataset and its uncompressed size is of about 280GB.
      \item The musiXmatch dataset has lyrics for almost 240k songs from the
            MSD. The reasons for the reduced size are: copyright, duplicates and
            instrumental tracks.
      \item The Last.fm dataset includes 94\% of the songs in the MSD, and in
            total around 500k have at least one tag.
    \end{itemize}

    \subsection{Downscaling}
    For this first milestone, we were tasked with selecting subset that can
    run in a single machine. In this case, the selection was straightforward
    since there is a 10k randomly sampled subset available in the MSD website.
    We considered this an appropriate size for the initial task.
    
    For the Last.fm dataset it was also possible to get the matching 10k
    subset but for the musiXmatch we needed to download the whole dataset
    with the 240k songs' lyrics.
    
    The sizes and formats of these subsets are listed in table
    \ref{tab:size_subset}.

    \begin{table}
      \center
      \caption{Sizes and formats for the data subsets}
      \begin{tabular}{|c|c|c|}
      \hline
      Dataset & Format & Size (GB) \\
      \hline
      MSD & Text files & 2.6 \\
      Last.fm & SQLite & 0.5 \\
      musiXmatch (Full) & SQLite & 2.3 \\
      \hline
      \end{tabular}
      \label{tab:size_subset}
    \end{table}

    \subsection{Preprocessing}
      In order to have a consistent access to all the information per track,
      we processed the input text files and SQLite databases and merged them
      into a single text file where each line is a JSON object with the
      track ID, its associated genres and the word frequencies for its
      lyrics.
      
      Here it was necessary to filter out the tracks without any genre information
      or less than 10 words in its lyrics, after this filtering the subset
      shrank to 1.4k songs and in its JSON representation it had a size of 1.1MB.
      
      \subsubsection{Finding out genres}    
        As mentioned in \ref{sec:data:subsec:description}, the Last.fm provides us
        with human-made tags, these tags however contain all sorts of information
        and are not focused on genres only. In order, to use these
        as ground truth for the experiments we cleaned the tags by comparing
        them against a comprehensive online list of genres
        \footnote{\url{http://www.musicgenreslist.com/}}.
        In order to capture possible misspellings we considered a tag equal to
        a genre in the list if their Damerau-Levenshtein distance
        \footnote{\url{http://en.wikipedia.org/wiki/Damerau-Levenshtein_distance}}
        does not exceed 1.
      
  \section{Design}
    \subsection{Algorithms}
      \subsection{K-Means clustering}
      \subsection{Locality Sensitive Hashing (LSH)}
    \subsection{Evaluation \& Metrics}
  \section{Implementation}
    \subsection{K-Means clustering in a single machine}
      \subsubsection{K-Means clustering in Hadoop}
    \subsection{LSH in a single machine}
      \subsubsection{LSH in Spark}
    \subsection{Web Service}
  \section{Results}
    \subsection{Quality of results}
      \subsubsection{K-Means}
      \subsubsection{LSH}
    \subsection{Performance \& Scalability}
  \bibliography{report_1}
\end{document}