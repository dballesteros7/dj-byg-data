\documentclass[11pt,a4paper,twoside]{scrartcl}
\usepackage[utf8]{inputenc}
\usepackage[english]{babel}
\usepackage{amsmath}
\usepackage{amsfonts}
\usepackage{amssymb}
\usepackage{graphicx}

\bibliographystyle{acm}
  
\author{Diego Ballesteros, Tracjhe Kralev, Tribhuvanesh Orekondy}
\title{DJ Tini Data}
\subtitle{The small scale implementation of DJ Byg Data}
\date{\today}

\begin{document}
  \maketitle
  \section{Introduction}
    The general aim of this project is to develop a system able to produce
    information out of data extracted from songs. This is the general
    aim of the research field known as Music Information Retrieval (MIR),
    this field has been active for quite some time \cite{nameThatTune:1993}
    and its applications can be seen in several commercial products such as
    Pandora or Spotify \cite{recommendation:2010}.

    A great portion of the activity in MIR deals with song meta-data and audio
    features \cite{McFee:2012:MSD:2187980.2188222}. However, in this project we
    focus on lyrics as the main source of data.
    The main task that we pursue is to answer the following question:
    
    "Is it possible to determine the genre of a song based solely on its
    lyrics?"
     
    In order to accomplish this, we first explored how to extract genre
    information from the songs' lyrics and then evaluated the quality of the
    judgments. Lastly we made the developed methods available as a web
    application where arbitrary lyrics can be input in order to obtain
    a genre prediction.
    
    The report is organized as follows, first we present the input datasets used
    for the task and their respective pre-processing and manipulation. We then
    proceed to briefly explain the computing and analysis techniques chosen for
    the task. In the third section we present details of the implementation of
    these techniques. Finally the obtained results and the performance of the
    algorithms is discussed.
  \section{Data}
    \subsection{Description}
    \subsection{Preprocessing}
    \subsection{Downscaling}
  \section{Design}
    \subsection{Algorithms}
      \subsection{K-Means clustering}
      \subsection{Locality Sensitive Hashing (LSH)}
    \subsection{Evaluation \& Metrics}
  \section{Implementation}
    \subsection{K-Means clustering in a single machine}
      \subsubsection{K-Means clustering in Hadoop}
    \subsection{LSH in a single machine}
      \subsubsection{LSH in Spark}
    \subsection{Web Service}
  \section{Results}
    \subsection{Quality of results}
      \subsubsection{K-Means}
      \subsubsection{LSH}
    \subsection{Performance \& Scalability}
  \bibliography{report_1}
\end{document}